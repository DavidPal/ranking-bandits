\documentclass{article}

\usepackage{fullpage,amssymb,amsthm,amsmath}
\usepackage{natbib}

\newcommand{\R}{\mathbb{R}}
\DeclareMathOperator*{\argmin}{argmin}

\begin{document}

\title{Ranking Bandits}
\author{R\'obert Busa-Fekete \and D\'avid P\'al}
\maketitle

\section{Introduction}

We consider an online ranking problem where users arrive one by one in an online
fashion and our goal is to present to each user an ordered list of documents. We
assume that for each user a certain subset documents is relevant. The cost
associated with each user is the position of the topmost relevant document. The
goal of the algorithm is to minimize the total cost.

After a ranking of documents is presented to a user, we receive certain
feedback. In this paper, we consider two feedback models. In the first feedback
model, which we call the \emph{full information} feedback model, we receive the
set of relevant documents for the user. In the second feedback model, which we
call the \emph{bandit information} feedback model, we receive only the position
of the topmost relevant document.

There exist a natural \emph{offline problem} for this online problem. Given sets
of relevant documents, one for each user, find a single ordered list of the
documents that minimizes the total cost. This problem was studied by
\cite{Feige-Lovasz-Tetali-2004}, who showed that a simple greedy algorithm
achieves approximation ratio $4$ and showed that achieving approximation ratio
$4-\epsilon$ is NP-hard; for generalizations see~\cite{Azar-Gamzu-Yin-2009}.

If each user's set of relevant documents has size $1$, the problem is a special
case of learning permutations \citep{Helmbold-Warmuth-2009,
Yasutake-Hatano-Kijima-Takimoto-Takeda-2011} and it is also a special case of
the online ranking problem studied by~\cite{Ailon-2014}. However, the general
problems are incomprable to ours as they use different cost functions.

\section{Formal Model}

Let $V$ be the set of all \emph{documents} and let $n = |V|$. A \emph{ranking}
of $V$ is a bijection $\pi:V \to \{1,2,\dots,n\}$. Given a ranking $\pi$, the
\emph{rank} of a document $v \in V$ is $\pi(v)$. There are $T$ users numbered
from $1,2,\dots,T$. For each user $t$ there is a non-empty set of \emph{relevant
documents} $U_t \subseteq V$.

The online problem proceeds in $T$ rounds. We distinguish between the
full information model and the bandit information feedback model.
In the full information feedback model, each round consists of three steps:
\begin{enumerate}
\item Choose a ranking $\pi_t$
\item Observe $U_t$
\item Incur cost $c(\pi_t, U_t) = \min_{v \in U_t} \pi_t(v)$
\end{enumerate}
In the bandit information feedback model, each round consists of two steps:
\begin{enumerate}
\item Choose a ranking $\pi_t$
\item Observe $c(\pi_t, U_t) = \min_{v \in U_t} \pi_t(v)$ and incur the same cost
\end{enumerate}


\section{Related work}

\begin{itemize}
\item \cite{Ailon-2014} $|U| = 1$
\item \cite{Helmbold-Warmuth-2009}, \cite{Yasutake-Hatano-Kijima-Takimoto-Takeda-2011} $|U| = 1$
\item \cite{Feige-Lovasz-Tetali-2004}, \cite{Azar-Gamzu-Yin-2009}
\item \cite{Radlinski-Kleinberg-Joachims-2008}
\item Online learning to rank based on Random projection \cite{Schuth-Oosterhuis-Whiteson-de-Rijke-2016}
\end{itemize}

\bibliography{biblio}
\bibliographystyle{plainnat}

\end{document}
