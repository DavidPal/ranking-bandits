\documentclass[12pt]{article}

\usepackage{fullpage,amssymb,amsthm,amsmath}
\usepackage{natbib}

\newtheorem{lemma}{Lemma}
\newtheorem{theorem}[lemma]{Theorem}

\newcommand{\N}{\mathbb{N}}
\newcommand{\R}{\mathbb{R}}
\newcommand{\indicator}{\mathbf{1}}
\DeclareMathOperator*{\argmin}{argmin}
\DeclareMathOperator*{\Exp}{\mathbf{E}}

\DeclareMathOperator{\cost}{\mathbf{cost}}
\DeclareMathOperator{\greedy}{\mathbf{greedy}}
\DeclareMathOperator{\online}{\mathbf{online}}
\DeclareMathOperator{\opt}{\mathbf{opt}}

\begin{document}

\title{Analysis of Greedy for $|U|=1$}
\author{D\'avid P\'al}
\maketitle

\section{Notation}

Let $N$ be a positive integer. Let $S_N$ be the set of all permutations of
$\{1,2,\dots,N\}$. In other words, $S_N$ is the set of all bijections from
$\{1,2,\dots,N\}$ to $\{1,2,\dots,N\}$. For a permutation $\pi \in S_N$ and a
vector $f \in \N^N$, we define the cost
$$
\cost(\pi,f) = \sum_{i=1}^N i \cdot f[\pi(i)] \; .
$$
Clearly, for any fixed $f \in \N^N$, a permutation $\pi$
is a minimizer of $\cost(\pi,f)$ if and only if
$$
f[\pi(1)] \ge f[\pi(2)] \ge \dots \ge f[\pi(N)] \; .
$$
For a given vector $f \in \N^N$, we denote by $\pi^*(f) = \argmin_{\pi \in S_N}
\cost(\pi, f)$. If there are multiple minimizers, we take the lexicographically
smallest one. Recall that a permutation $\pi \in S_N$ is lexicographically
smaller than a permutation $\pi' \in S_N$ if and only if there exists an index
$i \in \{1,2,\dots,N\}$ such that $\pi(1) = \pi'(1), \pi(2) = \pi'(2), \dots, \pi(i - 1) = \pi'(i -
1)$ and $\pi(i) < \pi'(i)$.


\section{Online Problem}

Let $e_i \in \N^N$ be a vector $i$-th vector of the standard basis.
That is, $e_i$ has coordinates $e_i[j] = \indicator\{i=j\}$
for $j=1,2,\dots,N$.

Let $a_1, a_2, \dots, a_T \in \{1,2,\dots,N\}$ be any sequence
and let $v_1, v_2, \dots, v_T$ where $v_t = e_{a_t}$ be the associated sequence
of standard basis vectors. Consider the partial sums
$$
f_t = \sum_{s=1}^t v_s \qquad \text{for $t=0,1,2,\dots,T$}.
$$

\emph{Online greedy algorithm} chooses in round $t$ the permutation $\pi^*(f_{t-1})$
and pays $\cost(\pi^*(f_{t-1}), v_t)$. Its total cost over $T$ rounds is
$$
\greedy = \sum_{t=1}^T \cost(\pi^*(f_{t-1}), v_t) \; .
$$

The cost of the optimal permutation in hindsight is
$$
\opt
= \min_{\pi \in S_N} \sum_{t=1}^T \cost(\pi, v_t)
= \min_{\pi \in S_N} \cost(\pi, f_T)
= \cost(\pi^*(f_T), f_T) \; .
$$

Our main result is that

\begin{theorem}
$$
\greedy \le 2 \cdot \opt + \, N^2 \; .
$$
\end{theorem}

\begin{proof}
Consider an algorithm that in round $t$ chooses permutation $\pi^*(f_t)$.
In other words, the algorithm has one step look ahead. Let
$$
\online = \sum_{t=1}^T \cost(\pi^*(f_t), v_t)
$$
be its cost. We will relate $\online$ to $\greedy$ and $\opt$.

First, we claim that
\begin{equation}
\label{equation:claim}
\online \le \opt \; .
\end{equation}
We prove the claim by induction on the number of rounds $T$. For that purpose,
we introduce subscripts to $\online$ and $\opt$. Namely, let $\online_T =
\sum_{t=1}^T \cost(\pi^*(f_t), v_t)$ and let $\opt_T = \min_{\pi \in S_N}
\sum_{t=1}^T \cost(\pi, v_t)$. In the base case $T=0$, we have $\online_T = 0$
and $\opt_T = 0$. For $T \ge 1$, the claim follows from the chain of
inequalities below:
\begin{align*}
\online_T
& = \sum_{t=1}^T \cost(\pi^*(f_t), v_t) \\
& = \online_{T-1} + \cost(\pi^*(f_T), v_T) \\
& \le \opt_{T-1} + \cost(\pi^*(f_T), v_T) \\
& = \cost(\pi^*(f_{T-1}), f_{T-1}) + \cost(\pi^*(f_T), v_T) \\
& = \min_{\pi \in S_N} \cost(\pi, f_{T-1}) + \cost(\pi^*(f_T), v_T) \\
& \le \cost(\pi^*(f_T), f_{T-1}) + \cost(\pi^*(f_T), v_T) \\
& = \opt_T \; .
\end{align*}
Above, the first inequality follows by induction hypothesis and
the second inequality is obvious.

The claim \eqref{equation:claim} implies that we can upper bound $\greedy$ as
$$
\greedy
= (\greedy - \online) + \online
\le (\greedy - \online) + \opt \; .
$$
Hence, we see that it remains to upper bound the difference
$$
\greedy - \online = \sum_{t=1}^T \cost(\pi^*(f_{t-1}), v_t) - \cost(\pi^*(f_t), v_t) \; .
$$
For notational convenience, let $\pi_t = \pi^*(f_t)$ and recall that $\pi_t$ is
the permutation that puts entries of $f_t$ in decreasing order (breaking ties
according to the index). Since $f_t = f_{t-1} + v_t$ and $v_t = e_{a_t}$, the
permutation $\pi_t$ can be obtained by simple modification of $\pi_{t-1}$.
Namely, let $i_t, j_t \in \{1,2,\dots,N\}$ be the indices such that $a_t =
\pi_{t-1}(i) = \pi_t(j)$. It easy to see that $j \le i$ and
$$
\pi^t(k) =
\begin{cases}
\pi^{t-1}(k) & \text{if $k < j$ or $k > i$} \; ,  \\
\pi^{t-1}(k-1) & \text{if $j < k \le i$} \; , \\
\pi^{t-1}(k) & \text{if $j = k$} \; . \\
\end{cases}
$$
In other words, the element $a_t$ moves from index $i_t$ to a smaller or equal
index $j_t$ and all other elements from $\{1,2,\dots,N\} \setminus \{a_t\}$ keep
their relative order. That is, elements between $j_t$ and $i_t$ shift by one and
remaining elements stay at their positions.

Since, $v_t = e_{a_t}$ is an indicator vector,
$$
\cost(\pi^*(f_{t-1}), v_t) - \cost(\pi^*(f_t), v_t) = i_t - j_t \; .
$$
Equivalently, $i_t - j_t$ can be expressed as the number of elements that are
``pushed'' by the element $a_t$. Formally,
\begin{align*}
i_t - j_t
& = |\{ k \in \{1,2,\dots,N\} ~:~ \pi_t(k) = \pi_{t-1}(k-1) \}| \\
& = |\{ b \in \{1,2,\dots,N\} ~:~ \pi_t^{-1}(b) = \pi_{t-1}^{-1}(b) + 1 \}| \; .
\end{align*}
We have
\begin{align*}
\greedy - \online
& = \sum_{t=1}^T \cost(\pi^*(f_{t-1}), v_t) - \cost(\pi^*(f_t), v_t) \\
& = \sum_{t=1}^T i_t - j_t \\
& = \sum_{t=1}^T \sum_{b=1}^N \indicator\{\pi_t^{-1}(b) = \pi_{t-1}^{-1}(b) + 1 \} \\
\end{align*}
Consider any elements $a,b \in \{1,2,\dots,N\}$ such that $a < b$. When $a$
pushes $b$ at time step $t$ it must be $f_t[a] = f_t[b]$. Since every time step
$a$ pushes $b$, its frequency $f_t[a]$ grows by one. Therefore the number of
pushes $a$ by $b$ is at most $\min\{f_T[a], f_T[b]\}$. Similarly, if $a > b$,
the number of pushes $a$ by $b$ is at most $\min\{f_T[a], f_T[b] + 1\}$.
Thus,
\begin{align*}
& \sum_{t=1}^T \sum_{a=1}^N \indicator\{\pi_t^{-1}(a) = \pi_{t-1}^{-1}(a) + 1 \} \\
& \le
\sum_{1 \le a < b \le N} \min\{f_T[a], f_T[b]\} +
\sum_{1 \le b < a \le N} \min\{f_T[a], f_T[b] + 1\} \\
& =
\sum_{1 \le a < b \le N} \sum_{f=1}^\infty \indicator\{f_T[a] \le f \wedge f_T[b] \le f\}
+ \sum_{1 \le b < a \le N} \sum_{f=1}^\infty \indicator\{f_T[a] \le f \wedge f_T[b] + 1 \le f\} \\
\end{align*}
\end{proof}
TODO
\end{document}
